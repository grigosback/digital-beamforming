\chapter{Conclusiones}\label{ch:conclusiones}
\chapterquote{He llegado hasta el fin, con los brazos cansados...}{Gustavo Cerati}

En este proyecto se realizó el estudio de distintas técnicas de implementación de un conformador de haz digital adaptativo, partiendo de una introducción teórica sobre los conceptos en los que se basa esta técnica, identificando los distintos tipos de conformadores de haz, definiendo el concepto de arreglo de antenas en fase y caracterizando sus distintos tipos, para luego definir el problema a resolver. A partir de aquí se realizó el estudio de las distintas técnicas de estimación de DOA, componente primordial para el funcionamiento del sistema a implementar. En este estudio se definió el modelo de muestras que se utilizó a lo largo de todo el proyecto y se realizó una introducción teórica sobre los conceptos algebraicos que permiten la descomposición del subespacio de muestras en subespacios de señal y ruido. La comprensión de esta técnica permitió iniciar con el análisis de dos algoritmos populares en lo que respecta a la estimación de parámetros de señales recibidas en arreglos de sensores: MUSIC y ESPRIT. Durante este estudio se desarrolló la teoría en la que se basa el funcionamiento de ambos algoritmos para finalmente realizar implementaciones de ambos que permitieron su comparación. En este estudio, MUSIC demostró ser un excelente algoritmo para introducirse en el análisis de técnicas de estimación paramétrica de señales debido a su intuitivo enfoque gráfico. Sin embargo, al momento de la implementación, ESPRIT demostró ser superior en lo que respecta a tiempos de ejecución, alcanzando niveles de error prácticamente idénticos. A partir de esto se decidió continuar con el algoritmo ESPRIT para el resto de la implementación.

Durante la realización de las simulaciones de los algoritmos de estimación de DOA implementados se observó que en situaciones particulares en las que las señales recibidas se encontraban muy correlacionadas se requería operar con una cantidad de muestras que volvía inviable la implementación de un sistema en tiempo real. A partir de aquí se logró dar con una técnica de muestreo aleatorio que dio excelentes resultados al momento de reducir la cantidad de muestras requeridas para realizar la estimación de DOA, reduciendo dicho número por encima de dos órdenes de magnitud. Luego de comprobar su correcto funcionamiento se logró dar con una explicación teórica de su eficacia, la cual se indica en el Capítulo $\ref{ch:randomsampling}$.

Para resolver el problema de estimación de cantidad de señales recibidas se observó que se podían utilizar técnicas de clasificación mediante aprendizaje automático. A partir de esto se pudo implementar un algoritmo que permitió realizar la clasificación de valores singulares, necesaria para realizar la estimación de cantidad de señales recibidas, el cual al ser comparado con el método de la máxima derivada, que consiste en hallar el umbral de separación entre valores singulares encontrando la máxima derivada de esta distribución discreta, mostró una mejora del 10\%, alcanzando una precisión de 97\%.

Una vez definidos los subsistemas que componen el conformador de haz se realizó un diseño de bloques, definiendo la función y las interfaces de cada uno de ellos, realizando un análisis cualitativo de las ventajas y desventajas que existen al implementar ciertas funciones en FPGA o en el PS. Finalmente se esboza una propuesta de diseño de bloques para FPGA para una futura implementación.

Por último se realizó una implementación en GNU Radio del sistema conformador de haz completo, validando su funcionamiento mediante simulaciones. Diseñando las interfaces necesarias, este software puede ser instalado en el PS de la placa de desarrollo para realizar pruebas iniciales cuando se construyan el resto de los sistemas que complementan al conformador de haz (sistema de adquisición y arreglo de antenas).

Como cierre de este trabajo se esbozaron algunas propuestas de estudio e implementación a futuro con el objetivo de motivar la continuación de este proyecto analizando alternativas que pueden llegar a conseguir resultados aún mejores a los obtenidos.