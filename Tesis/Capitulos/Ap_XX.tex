\chapter{Obtención de ángulos de arribo en ESPRIT}\label{AP:esprit_angles}
%\chapterquote{Negociemos Don Inodoro}{Fernando de la R\'{u}a, 2001}
%\graphicspath{{figs/}}
%%%%%%%%%%%%%%%%%%%%%%%%%%%%%%%%%%%%%%%%%%%%%%%%%%%%%%%%%%%%%%%%%%%%%%%%

Considerando una elección de subarreglos como el de la Figura \ref{fig:doaest_esprit2d} y el sistema de coordenadas definido en la Figura \ref{fig:beamforming_ura}, los fasores $\phi_x$ y $\phi_y$ obtenidos mediante ESPRIT valen:

\begin{equation}
    \begin{aligned}
        \phi_x & = e^{-jk\delta\cos\theta \cos\varphi}  \\
        \phi_y & = e^{-jk\delta\cos\theta \sin \varphi}
    \end{aligned}
\end{equation}

Quedándonos con la fase de estas expresiones tenemos:

\begin{equation}
    \begin{aligned}
        \angle\phi_x & = k\delta\cos\theta \cos\varphi  \\
        \angle\phi_y & = k\delta\cos\theta \sin \varphi
    \end{aligned}
\end{equation}

Dividiendo $\angle\phi_y$ con $\angle\phi_x$ se tiene:

\begin{equation}
    \frac{\angle\phi_y}{\angle\phi_x} = \frac{\sin\varphi}{\cos \varphi} = \tan\varphi
\end{equation}
por ende de aquí puede obtenerse $\varphi$ haciendo:
\begin{equation}
    \varphi = \arctan2(\angle\phi_y,\angle\phi_x)
\end{equation}

Para obtener $\theta$ se procede haciendo:
\begin{align}
    (\angle\phi_x)^2 + (\angle\phi_y)^2 & = k^2 \delta^2 \cos^2 \theta \underbrace{\left(\cos^2 \varphi + \sin^2 \varphi \right)}_1\nonumber \\
    (\angle\phi_x)^2 + (\angle\phi_y)^2 & = k^2 \delta^2 \cos^2 \theta\nonumber                                                              \\
    \cos^2 \theta                       & = \frac{(\angle\phi_x)^2 + (\angle\phi_y)^2}{k^2 \delta^2}\nonumber                                \\
    \cos \theta                         & = \sqrt{\frac{(\angle\phi_x)^2 + (\angle\phi_y)^2}{k^2 \delta^2}}\nonumber                         \\
    \theta                              & = \arccos\left(\sqrt{\frac{(\angle\phi_x)^2 + (\angle\phi_y)^2}{k^2 \delta^2}} \right)
\end{align}

%%% Local Variables: 
%%% mode: latex
%%% TeX-master: "template"
%%% End: 
