\begin{resumen}%
    En el siguiente trabajo se realiza el análisis de distintas técnicas de conformación digital de haces mediante el empleo de un arreglo de antenas para la recepción, haciendo un hincapié inicial en el estudio de los algoritmos de estimación de dirección de arribo MUSIC y ESPRIT. Luego se introduce la técnica de muestreo aleatorio, la cual permite realizar estimaciones de los parámetros de las señales recibidas reduciendo por encima de dos órdenes de magnitud la cantidad de muestras necesarias con respecto al muestreo ideal. Seguido a esto se realiza el análisis de técnicas de estimación de cantidad de señales recibidas, introduciendo un método de estimación que utiliza el algoritmo de aprendizaje automático de máquinas de vectores de soporte. Por último se analizan propuestas de implementación del sistema conformador de haz en una placa de desarrollo CIAA-ACC con capacidad de distribuir funciones entre una FPGA y un microprocesador, para finalmente mostrar una implementación de todo el sistema diseñado utilizando GNU Radio, validando su funcionamiento con simulaciones.
\end{resumen}

\begin{abstract}%
    In the following work, the analysis of different digital beamforming techniques is performed considering an antenna array for the signals reception, with an initial emphasis on the study of MUSIC and ESPRIT direction of arrival estimation algorithms. A random sampling technique is introduced, which allows received signals parameter estimation with a reduction in the number of samples required by over two orders of magnitude with respect to uniform sampling. Also, this work presents an analysis of techniques for estimating the number of received signals, followed by the implementation of an estimation method that uses a Support Vector Machine algorithm. Finally, proposals for the implementation of the beamformer system are analyzed on a CIAA-ACC development board with the ability to distribute functions between a FPGA and a microprocessor, and an implementation of the entire system designed using GNU Radio is also presented, validating its operation with simulations.
\end{abstract}